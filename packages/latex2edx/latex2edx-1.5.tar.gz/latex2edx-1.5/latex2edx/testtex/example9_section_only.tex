%
% File:   example9.tex
%
% 

\documentclass[12pt]{article}

\usepackage{edXpsl}	% edX

%%%%%%%%%%%%%%%%%%%%%%%%%%%%%%%%%%%%%%%%%%%%%%%%%%%%%%%%%%%%%%%%%%%%%%%%%%%%%
\begin{document}

\begin{edXsection}{A second section}[due="2016-11-22" graded=true]

\begin{edXproblem}{Example problem with hints}{url_name="p1"}
 
This is an edXmath example.

\begin{edXmath}
\begin{eqnarray}
S(\rho) &=&  -\lambda_{1} \log \lambda_{1} -\lambda_{2} \log \lambda_{2} \\
        &=&  H((1+r)/2)
\end{eqnarray}
\end{edXmath}

This is text after the edXmath.

Contrast that with

\begin{eqnarray}
S(\rho) &=&  -\lambda_{1} \log \lambda_{1} -\lambda_{2} \log \lambda_{2} \\
        &=&  H((1+r)/2)
\end{eqnarray}

which plastex mangles into a table, that still does render ok.

\end{edXproblem}

%%%%%%%%%%%%%%%%%%%%%%%%%%%%%%%%%%%%%%%%

\end{edXsection}

%%%%%%%%%%%%%%%%%%%%%%%%%%%%%%%%%%%%%%%%%%%%%%%%%%%%%%%%%%%%%%%%%%%%%%%%%%%%%

\end{document}
